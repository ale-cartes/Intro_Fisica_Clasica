\documentclass[letterpaper,11pt]{article}
\oddsidemargin -1.0cm \textwidth 17.5cm

\usepackage[utf8]{inputenc}
\usepackage[activeacute,spanish]{babel}
\usepackage{amsfonts,setspace}
\usepackage{amsmath}
\usepackage{amssymb, amsmath, amsthm}
\usepackage{comment}
\usepackage{amssymb}
\usepackage{dsfont}
\usepackage{anysize}
\usepackage{multicol}
\usepackage{enumerate}
\usepackage{graphicx}
\usepackage[left=1.5cm,top=2cm,right=1.5cm, bottom=1.7cm]{geometry}
\setlength\headheight{1.5em} 
\usepackage{fancyhdr}
\usepackage{multicol}
\usepackage{hyperref}
\usepackage{wrapfig}
\pagestyle{fancy}
\fancyhf{}
\renewcommand{\labelenumi}{\normalsize\bfseries P\arabic{enumi}.}
\renewcommand{\labelenumii}{\normalsize\bfseries (\alph{enumii})}
\renewcommand{\labelenumiii}{\normalsize\bfseries \roman{enumiii})}

\begin{document}

\fancyhead[L]{\itshape{Facultad de Ciencias F\'isicas y Matem\'aticas}}
\fancyhead[R]{\itshape{Universidad de Chile}}

\begin{minipage}{11.5cm}
    \begin{flushleft}
        \hspace*{-0.6cm}\textbf{FI1000-5 Introducción a la Física Clásica}\\
        \hspace*{-0.6cm}\textbf{Profesora:} Paulina Lira\\
        \hspace*{-0.6cm}\textbf{Auxiliares:} Alejandro Silva, Felipe Kaschel, Juan Cristóbal Castro\\
    \end{flushleft}
\end{minipage}

\begin{picture}(2,3)
    \put(366,-4){\includegraphics[scale=0.9]{2020-1/Imágenes/logo/dfi-fcfm.pdf}}
\end{picture}

\begin{center}
	\LARGE \bf Ejercicio \#8\\
\end{center}

\vspace{-1cm}
\begin{enumerate}\setlength{\itemsep}{0.4cm}

\rfoot[]{pág. \thepage}

\item[]

\item Se tiene el siguiente dispositivo de la figura para lanzar bolitas. Si el resorte tiene una constante de elasticidad $\mathbf{k}$ y la bolita una masa $\mathbf{m}$, encuentre la velocidad de la bolita al momento que se suelta del émbolo si el resorte se ha comprimido en un largo $\mathbf{x}$. ¿Hasta qué altura habrá llegado la bolita al momento de detenerse? Considere una superficie lo suficientemente larga y sin roce.

\begin{figure}[h!]
    \centering
    \includegraphics[scale = 0.6]{2020-1/Imágenes/ejercicios/dispositivo.pdf}
    \caption{Dispositivo lanza bolitas}
\end{figure}

\end{enumerate}
\end{document}
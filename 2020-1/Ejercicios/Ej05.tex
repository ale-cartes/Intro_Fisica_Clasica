\documentclass[letterpaper,11pt]{article}
\oddsidemargin -1.0cm \textwidth 17.5cm

\usepackage[utf8]{inputenc}
\usepackage[activeacute,spanish]{babel}
\usepackage{amsfonts,setspace}
\usepackage{amsmath}
\usepackage{amssymb, amsmath, amsthm}
\usepackage{comment}
\usepackage{amssymb}
\usepackage{dsfont}
\usepackage{anysize}
\usepackage{multicol}
\usepackage{enumerate}
\usepackage{graphicx}
\usepackage[left=1.5cm,top=2cm,right=1.5cm, bottom=1.7cm]{geometry}
\setlength\headheight{1.5em} 
\usepackage{fancyhdr}
\usepackage{multicol}
\usepackage{hyperref}
\usepackage{wrapfig}
\pagestyle{fancy}
\fancyhf{}
\renewcommand{\labelenumi}{\normalsize\bfseries P\arabic{enumi}.}
\renewcommand{\labelenumii}{\normalsize\bfseries (\alph{enumii})}
\renewcommand{\labelenumiii}{\normalsize\bfseries \roman{enumiii})}

\begin{document}

\fancyhead[L]{\itshape{Facultad de Ciencias F\'isicas y Matem\'aticas}}
\fancyhead[R]{\itshape{Universidad de Chile}}

\begin{minipage}{11.5cm}
    \begin{flushleft}
        \hspace*{-0.6cm}\textbf{FI1000-5 Introducción a la Física Clásica}\\
        \hspace*{-0.6cm}\textbf{Profesora:} Paulina Lira\\
        \hspace*{-0.6cm}\textbf{Auxiliares:} Alejandro Silva, Felipe Kaschel, Juan Cristobal Castro\\
    \end{flushleft}
\end{minipage}

\begin{picture}(2,3)
    \put(405,-5){\includegraphics[scale=1.25]{2020-1/Imágenes/logo/fcfm2.pdf}}
\end{picture}

\begin{center}
	\LARGE \bf Ejercicio \#5   \\
\end{center}

\vspace{-1cm}
\begin{enumerate}\setlength{\itemsep}{0.4cm}

\rfoot[]{pág. \thepage}

\item[]

\item La última atracción de Fantasilandia es un tambor de radio $R$ que gira sobre su eje suficientemente rápido de modo que una persona parada en su interior es mantenida erguida contra las paredes cuando la parte inferior del tambor se deja caer.\\
Muestre que el periodo máximo de giro del tambor, para evitar que la gente se caiga, debe ser igual a 
\[T = \sqrt{\frac{4\pi^2R\mu}{g}}\]

Además encuentre el valor numérico de $T$ si $R = 4$ m, el coeficiente de roce estático es $\mu_s = 0.4$ y el coeficiente de roce cinético es $\mu_k = 0.2$. Argumente su elección.

\begin{figure}[h!]
    \centering
    \includegraphics[scale = 0.5]{2020-1/Imágenes/ejercicios/Ejercicio_5_cropped.pdf}
    \caption{Juegazo de Fantasilandia}
\end{figure}
\end{enumerate}
\end{document}
\documentclass[letterpaper,11pt]{article}
\oddsidemargin -1.0cm \textwidth 17.5cm

\usepackage[utf8]{inputenc}
\usepackage[activeacute,spanish]{babel}
\usepackage{amsfonts,setspace}
\usepackage{amsmath}
\usepackage{amssymb, amsmath, amsthm}
\usepackage{comment}
\usepackage{amssymb}
\usepackage{dsfont}
\usepackage{anysize}
\usepackage{multicol}
\usepackage{enumerate}
\usepackage{graphicx}
\usepackage[left=1.5cm,top=2cm,right=1.5cm, bottom=1.7cm]{geometry}
\setlength\headheight{1.5em} 
\usepackage{fancyhdr}
\usepackage{multicol}
\usepackage{hyperref}
\usepackage{wrapfig}
\pagestyle{fancy}
\fancyhf{}
\renewcommand{\labelenumi}{\normalsize\bfseries P\arabic{enumi}.}
\renewcommand{\labelenumii}{\normalsize\bfseries (\alph{enumii})}
\renewcommand{\labelenumiii}{\normalsize\bfseries \roman{enumiii})}

\begin{document}

\fancyhead[L]{\itshape{Facultad de Ciencias F\'isicas y Matem\'aticas}}
\fancyhead[R]{\itshape{Universidad de Chile}}

\begin{minipage}{11.5cm}
    \begin{flushleft}
        \hspace*{-0.6cm}\textbf{FI1000-5 Introducción a la Física Clásica}\\
        \hspace*{-0.6cm}\textbf{Profesora:} Paulina Lira\\
        \hspace*{-0.6cm}\textbf{Auxiliares:} Alejandro Silva, Felipe Kaschel, Juan Cristóbal Castro\\
    \end{flushleft}
\end{minipage}

\begin{picture}(2,3)
    \put(366,-4){\includegraphics[scale=0.9]{2020-1/Imágenes/logo/dfi-fcfm.pdf}}
\end{picture}

\begin{center}
	\LARGE \bf Auxiliar Extra C1   \\
\end{center}

\vspace{-1cm}
\begin{enumerate}\setlength{\itemsep}{0.4cm}

\rfoot[]{pág. \thepage}

\item[]

\item Una bala es disparada con velocidad $v_b$ hacia una lata que gira sobre su eje de simetría. Determine la frecuencia en rpm a la que gira la lata tal que la bala salga por el mismo agujero que hizo al entrar, sabiendo que la bala reduce en 1/3 su velocidad debido al impacto con la lata.

\item Se lanzan los proyectiles 1 y 2 con velocidad $v_0$ y ángulos $\alpha$ y $\beta$ respectivamente. Se sabe que ambos recorren una distancia D en la horizontal. Encuentre una relación entre el tiempo que tardan 1 y 2 en caer al suelo.

 \item Un ciclista pedalea a una frecuencia f. Si el radio del disco es D, el del piñón es P y el de la rueda es R.
 \begin{enumerate}
     \item Determine la velocidad a la que avanza el ciclista.
     \item Debido a una bajada la velocidad del ciclista es el triple de la velocidad anterior. Encuentre su nueva velocidad.
 \end{enumerate}
 
\item ( Apunte Hugo Arellano, Ejercicio 8) Una linterna asciende verticalmente con rapidez constante $u$ iluminando de forma cónica un área circular sobre el piso. Al mismo tiempo un razón sale de su casa en un trayecto rectilíneo que atraviesa diametralmente el área iluminada. Inicialmente el ratón sale de su casa y la linterna comienza a subir desde el piso a una distancia D del ratón. El cono de iluminación está caracterizado por un 
ángulo directriz $\phi$. Calcule el lapso que el ratón es iluminado por la linterna.

\item Tres bloques de igual masa m se posan sobre un plano horizontal. El coeficiente de roce entre cada bloque y el piso es $\mu$. Los dos primeros bloques se unen mediante una cuerda ideal mientras que los dos últimos se unen mediante un resorte de constante elástica k. Una fuerza horizontal aplicada al primer bloque hace que los tres bloques se muevan manteniendo la elongación del resorte constante e igual a $\Delta$. Determine la magnitud de la fuerza aplicada.
\end{enumerate}
\end{document}
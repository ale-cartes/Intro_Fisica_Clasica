\documentclass[letterpaper,11pt]{article}
\oddsidemargin -1.0cm \textwidth 17.5cm

\usepackage[utf8]{inputenc}
\usepackage[activeacute,spanish, es-lcroman]{babel}
\decimalpoint
\usepackage{amsfonts,setspace}
\usepackage{amsmath}
\usepackage{amssymb, amsmath, amsthm}
\usepackage{comment}
\usepackage{float}
\usepackage{amssymb}
\usepackage{dsfont}
\usepackage{anysize}
\usepackage{multicol}
\usepackage{enumerate}
\usepackage{graphicx}
\usepackage[left=1.5cm,top=2cm,right=1.5cm, bottom=1.7cm]{geometry}
\setlength\headheight{1.5em} 
\usepackage{fancyhdr}
\usepackage{multicol}
\usepackage{hyperref}
\usepackage{wrapfig}
\usepackage{subcaption}
\usepackage{siunitx}
\usepackage{cancel}
\usepackage{mdwlist}
\usepackage{svg}
\pagestyle{fancy}
\fancyhf{}
\renewcommand{\labelenumi}{\normalsize\bfseries P\arabic{enumi}.}
\renewcommand{\labelenumii}{\normalsize\bfseries (\alph{enumii})}
\renewcommand{\labelenumiii}{\normalsize\bfseries \roman{enumiii})}


\begin{document}

\fancyhead[L]{\itshape{Facultad de Ciencias F\'isicas y Matem\'aticas}}
\fancyhead[R]{\itshape{Universidad de Chile}}

\begin{minipage}{11.5cm}
    \begin{flushleft}
        \hspace*{-0.6cm}\textbf{FI1000-1 Introducción a la Física Clásica}\\
        \hspace*{-0.6cm}\textbf{Profesora:} Jocelyn Dunstan\\
        \hspace*{-0.6cm}\textbf{Auxiliar:} Alejandro Silva\\
        \hspace*{-0.6cm}\textbf{Ayudantes:} Macarena Muñoz \& Catalina Vargas\\
    \end{flushleft}
\end{minipage}

\begin{picture}(2,3)
    \svgpath{../}  % descomentar si se agrega a carpeta "auxiliares"/"ejercicios"
    \put(366, 10){\includesvg[scale=0.31]{img/dfi.svg}}
\end{picture}

\begin{center}
	\LARGE\textbf{Auxiliar \#5}\\
	\Large{MCUA y Mov. Relativo}
\end{center}

\vspace{-1.0cm}
\begin{enumerate}\setlength{\itemsep}{0.4cm}

\rfoot[]{pág. \thepage}

\item[]

\item Un disco con un agujero a una distancia $R$ del centro está inicialmente en reposo. De repente, se lanza un proyectil verticalmente con velocidad $v_0$ de manera que pase por este agujero. Para que el proyectil logre caer en el mismo agujero, el disco comienza a girar con aceleración angular $\alpha$ constante inmediatamente cuando el proyectil pasa por este. Determine, en el instante en que el proyectil vuelve a pasar por el agujero:
\begin{enumerate}
    \item $\alpha$ y la velocidad angular del disco
    
    \item la velocidad tangencial y el tamaño de la aceleración total que posee el agujero
\end{enumerate}

\begin{figure}[H]
    \centering
    \svgpath{../img/aux5}
    \includesvg[width=0.5\linewidth]{disco.svg}
\end{figure}

\item Considere la situación en la que usted va en un auto en una autopista a cierta velocidad $v_0$. En la autopista se encuentra con dos autos, uno que se mueve con velocidad $u_0 > v_0$ y otro que se mueve con velocidad $w_0 < v_0$. Determine las velocidades relativas a usted que poseen los autos. 

\item Un auto avanza bajo la lluvia a una velocidad constante $v_0$. Las gotas de lluvia caen verticalmente respecto a la Tierra a una velocidad constante $u_0$. Debido al movimiento, vistas desde la ventana del auto, las gotas de lluvia parecen caer formando un ángulo $\alpha$ respecto a la vertical. Determine la velocidad $u_0$ con que caen las gotas de lluvia.

\end{enumerate}
\end{document}

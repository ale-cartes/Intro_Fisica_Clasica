\documentclass[letterpaper,11pt]{article}
\oddsidemargin -1.0cm \textwidth 17.5cm

\usepackage[utf8]{inputenc}
\usepackage[activeacute,spanish, es-lcroman]{babel}
\decimalpoint
\usepackage{amsfonts,setspace}
\usepackage{amsmath}
\usepackage{amssymb, amsmath, amsthm}
\usepackage{comment}
\usepackage{float}
\usepackage{amssymb}
\usepackage{dsfont}
\usepackage{anysize}
\usepackage{multicol}
\usepackage{enumerate}
\usepackage{graphicx}
\usepackage[left=1.5cm,top=2cm,right=1.5cm, bottom=1.7cm]{geometry}
\setlength\headheight{1.5em} 
\usepackage{fancyhdr}
\usepackage{multicol}
\usepackage{hyperref}
\usepackage{wrapfig}
\usepackage{subcaption}
\usepackage{siunitx}
\usepackage{cancel}
\usepackage{mdwlist}
\usepackage{svg}
\pagestyle{fancy}
\fancyhf{}
\renewcommand{\labelenumi}{\normalsize\bfseries P\arabic{enumi}.}
\renewcommand{\labelenumii}{\normalsize\bfseries (\alph{enumii})}
\renewcommand{\labelenumiii}{\normalsize\bfseries \roman{enumiii})}


\begin{document}

\fancyhead[L]{\itshape{Facultad de Ciencias F\'isicas y Matem\'aticas}}
\fancyhead[R]{\itshape{Universidad de Chile}}

\begin{minipage}{11.5cm}
    \begin{flushleft}
        \hspace*{-0.6cm}\textbf{FI1000-1 Introducción a la Física Clásica}\\
        \hspace*{-0.6cm}\textbf{Profesora:} Jocelyn Dunstan\\
        \hspace*{-0.6cm}\textbf{Auxiliar:} Alejandro Silva\\
        \hspace*{-0.6cm}\textbf{Ayudantes:} Macarena Muñoz \& Catalina Vargas\\
    \end{flushleft}
\end{minipage}

\begin{picture}(2,3)
    \svgpath{../}  % descomentar si se agrega a carpeta "auxiliares"/"ejercicios"
    \put(366, 10){\includesvg[scale=0.31]{img/dfi.svg}}
\end{picture}

\begin{center}
	\LARGE\textbf{Auxiliar \#2}\\
	\Large{Cinemática en 1D}
\end{center}

\vspace{-1cm}
\begin{enumerate}\setlength{\itemsep}{0.4cm}

\rfoot[]{pág. \thepage}

\item[]

\item Una partícula se mueve a lo largo del eje $x$. En el tiempo $t=0$, la partícula se encuentra en $x=0$. La velocidad de la partícula cambia en función del tiempo como se muestra en el gráfico. Determine:

{
    \begin{multicols}{2}
        \begin{enumerate}
            
            \item La posición en $t=\SI{1}{\second}$
            
            \item La aceleración en $t=\SI{2}{\second}$
            
            \item La posición en $t=\SI{4}{\second}$
            
            \item La velocidad promedio entre $t=\SI{0}{\second}$ y $t=\SI{3}{\second}$
            
            \item La aceleración instantánea en $t=\SI{1}{\second}$ y en $t=\SI{3}{\second}$. ¿Es posible este movimiento físicamente?
        \end{enumerate}
        
        \columnbreak
        
        \begin{figure}[H]
            \centering
            \svgpath{../img/aux2}
            \includesvg[width=0.77\linewidth]{plot.svg}
        \end{figure}
    \end{multicols}
}

\item Dos vehículos parten de un mismo lugar y deben recorrer una distancia total $L$. El vehículo $A$ parte del reposo y recorrer la mitad del camino ($3L/4$) con aceleración constante $a_0$, y luego mantiene la velocidad final alcanzada, recorriendo la segunda mitad del camino con velocidad constante. Si el vehículo $B$ hace todo el recorrido a velocidad constante, determine a qué velocidad debe viajar para llegar al mismo instante que $A$ al final del trayecto.

\begin{figure}[H]
    \centering
    \svgpath{../img/aux2} 
    \includesvg[width=0.65\linewidth]{auto.svg}
\end{figure}

\item

{
    \begin{multicols}{2}
        Una bola de acero se deja caer desde el techo de un edificio. Un observador parado frente a una ventana de altura $h$ nota que la bola cruza la ventana en $\tau$ segundos. La bola continua cayendo hasta chocar en forma completamente elástica con el piso (es decir, el módulo de su velocidad no cambia) y reaparece en la parte baja de la ventana $\tau_0$ segundos después. Demuestre que la altura del edificio está dada por la siguiente expresión:
        
        \centering{$H = \cfrac{g}{8}\left(\tau_0 + \tau + \cfrac{2h}{\tau g}\right)^2$}
        
        \columnbreak
        
        \begin{figure}[H]
            \centering
            \svgpath{../img/aux2}
            \includesvg[width=0.6\linewidth]{sus.svg}
        \end{figure}
        
    \end{multicols}
}

\item \textbf{[Propuesto]} Dos autos (A y B) avanzan juntos por una calle, ambos con velocidad constante $v$. Cuando ambos autos están a distancia $L$ de un cruce se prende la luz amarilla del semáforo. El auto $A$ empieza a frenar con aceleración constante a modo de detenerse justo en el cruce. En tanto, el auto $B$ mantiene su velocidad. Transcurrido un tiempo $t_1$ desde que la luz cambió a amarilla, el semáforo cambia a rojo y entonces el auto $B$ empieza a frenar con aceleración constante para detenerse justo en el cruce (el auto $A$ sigue con aceleración que ya traía). Ambos autos se detienen exactamente en el mismo lugar.

\begin{enumerate}
    \item Muestre que es imposible que se detengan al mismo tiempo
    
    \item Grafique la posición y la velocidad de ambos autos en función del tiempo
\end{enumerate}

% Para imágenes vectoriales -> el texto tiene que estar en LaTeX
% \begin{figure}[htbp]
%   \centering
%   \svgpath{../Imagenes/ejercicios}  -> .. irse pa'trás 
%   \includesvg{ej5.svg}
% \end{figure}

\end{enumerate}
\end{document}

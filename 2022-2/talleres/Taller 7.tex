\documentclass[letterpaper,11pt]{article}
\oddsidemargin -1.0cm \textwidth 17.5cm

\usepackage[utf8]{inputenc}
\usepackage[activeacute,spanish, es-lcroman]{babel}
\decimalpoint
\usepackage{amsfonts,setspace}
\usepackage{amsmath}
\usepackage{amssymb, amsmath, amsthm}
\usepackage{comment}
\usepackage{float}
\usepackage{amssymb}
\usepackage{dsfont}
\usepackage{anysize}
\usepackage{multicol}
\usepackage{enumerate}
\usepackage{graphicx}
\usepackage[left=1.5cm,top=2cm,right=1.5cm, bottom=1.7cm]{geometry}
\setlength\headheight{1.5em} 
\usepackage{fancyhdr}
\usepackage{multicol}
\usepackage{hyperref}
\usepackage{wrapfig}
\usepackage{subcaption}
\usepackage{siunitx}
\usepackage{cancel}
\usepackage{mdwlist}
\usepackage{svg}
\pagestyle{fancy}
\fancyhf{}
\renewcommand{\labelenumi}{\normalsize\bfseries P\arabic{enumi}.}
\renewcommand{\labelenumii}{\normalsize\bfseries (\alph{enumii})}
\renewcommand{\labelenumiii}{\normalsize\bfseries \roman{enumiii})}


\begin{document}

\fancyhead[L]{\itshape{Facultad de Ciencias F\'isicas y Matem\'aticas}}
\fancyhead[R]{\itshape{Universidad de Chile}}

\begin{minipage}{11.5cm}
    \begin{flushleft}
        \hspace*{-0.6cm}\textbf{FI1000-1 Introducción a la Física Clásica}\\
        \hspace*{-0.6cm}\textbf{Profesor:} Ignacio Bordeu\\
        \hspace*{-0.6cm}\textbf{Auxiliares:} Javier Cubillos \& Berenice Muruaga\\
        \hspace*{-0.6cm}\textbf{Auxiliares taller:} Pablo González \& Alejandro Cartes\\
        \hspace*{-0.6cm}\textbf{Ayudante:} Amaru Moya\\
    \end{flushleft}
\end{minipage}

\begin{picture}(2,3)
    \put(366, 10){\includegraphics[scale=0.9]{2020-1/Imágenes/logo/dfi-fcfm.pdf}}
\end{picture}

\begin{center}
	\LARGE\textbf{Taller \#7}\\
	\Large{Trabajo y Energía}
\end{center}

\vspace{-1cm}
\begin{enumerate}\setlength{\itemsep}{0.4cm}\addtocounter{enumi}{0}

\rfoot[]{pág. \thepage}

\item[]

\item Un bloque de masa $m$ se desliza por un plano inclinado sin roce de largo $L$ con un ángulo de inclinación~$\theta$ con respecto a la horizontal. Al final de este plano, el bloque se mueve por una superficie áspera con coeficiente de roce cinético $\mu_c$. Determine la distancia $D$ a la cual el bloque se detendrá en esta superficie áspera.

\begin{figure}[htbp]
  \centering
  \svgpath{../../2021-1/Imagenes/ejercicios}
  \includesvg[width=0.7\linewidth]{ej7.svg}
\end{figure}

\item Como se muestra en la figura, se tiene una masa atada al extremo de una cuerda de longitud $L$ que se coloca en posición horizontal y se suelta. Cuando la masa está en el punto más bajo, la cuerda choca con una delgada clavija ubicada a una distancia $R$ por encima de dicho punto. Demuestre que $R$ debe ser menor que $2L/5$ para que la masa describa un círculo entero alrededor de la clavija.

\begin{figure}[H]
  \centering
  \svgpath{../../2022-1//img/ejercicios}
  \includesvg[width=0.4\linewidth]{ej6.svg}
\end{figure}

% Para imágenes vectoriales -> el texto tiene que estar en LaTeX
% \begin{figure}[htbp]
%   \centering
%   \svgpath{../Imagenes/ejercicios}  -> .. irse pa'trás 
%   \includesvg{ej5.svg}
% \end{figure}

\end{enumerate}
\end{document}

\documentclass[letterpaper,11pt]{article}
\oddsidemargin -1.0cm \textwidth 17.5cm

\usepackage[utf8]{inputenc}
\usepackage[activeacute,spanish, es-lcroman]{babel}
\decimalpoint
\usepackage{amsfonts,setspace}
\usepackage{amsmath}
\usepackage{amssymb, amsmath, amsthm}
\usepackage{comment}
\usepackage{float}
\usepackage{amssymb}
\usepackage{dsfont}
\usepackage{anysize}
\usepackage{multicol}
\usepackage{enumerate}
\usepackage{graphicx}
\usepackage[left=1.5cm,top=2cm,right=1.5cm, bottom=1.7cm]{geometry}
\setlength\headheight{1.5em} 
\usepackage{fancyhdr}
\usepackage{multicol}
\usepackage{hyperref}
\usepackage{wrapfig}
\usepackage{subcaption}
\usepackage{siunitx}
\usepackage{cancel}
\usepackage{mdwlist}
\usepackage{svg}
\pagestyle{fancy}
\fancyhf{}
\renewcommand{\labelenumi}{\normalsize\bfseries P\arabic{enumi}.}
\renewcommand{\labelenumii}{\normalsize\bfseries (\alph{enumii})}
\renewcommand{\labelenumiii}{\normalsize\bfseries \roman{enumiii})}


\begin{document}

\fancyhead[L]{\itshape{Facultad de Ciencias F\'isicas y Matem\'aticas}}
\fancyhead[R]{\itshape{Universidad de Chile}}

\begin{minipage}{11.5cm}
    \begin{flushleft}
        \hspace*{-0.6cm}\textbf{FI1000-1 Introducción a la Física Clásica}\\
        \hspace*{-0.6cm}\textbf{Profesor:} Ignacio Bordeu\\
        \hspace*{-0.6cm}\textbf{Auxiliares:} Alejandro Cartes \& Simón Yáñez\\
        \hspace*{-0.6cm}\textbf{Ayudante:} Javier Cubillos\\
    \end{flushleft}
\end{minipage}

\begin{picture}(2,3)
    \put(366, 10){\includegraphics[scale=0.9]{2020-1/Imágenes/logo/dfi-fcfm.pdf}}
\end{picture}

\begin{center}
	\LARGE\textbf{Auxiliar \#4}\\
	\Large{MCUA y Leyes de Newton}
\end{center}

\vspace{-1cm}
\begin{enumerate}\setlength{\itemsep}{0.4cm}

\rfoot[]{pág. \thepage}

\item[]

\item Dos masas $m_1$ y $m_2$ describen dos movimientos circunferenciales concéntricos uniformemente acelerados por $\alpha _1$ y $\alpha _2$ respectivamente. Existe un ángulo $\theta _o$ entre $m_1$ y $m_2$ al iniciar su movimiento desde el reposo. Las circunferencias que describen poseen un radio $R$ y $r$ como se describe en la figura.
%(y) 
\begin{figure}[H]
    \centering
    \svgpath{../../2023-1/img/aux_4}
    \hspace{1em}
    \includesvg[width=0.8\textwidth]{aux_4_p1.svg}
\end{figure}

\begin{enumerate}
    \item Calcule el tiempo que toma dar $N$ revoluciones para cada masa. ¿Cómo se comporta el tiempo de revolución gráficamente?
    \item Encuentre la relación que cumplen las aceleraciones si $m_1$ hace 3 revoluciones por cada 2 de $m_2$
    \item Considerando $\theta _o = \frac{7\pi}{8}$. Exprese el ángulo $\beta$ que forma $m_2$ respecto a $m_1$ para toda revolución de $m_1$. Evalué en $N=9$
    \item Si ambos ejes soportan la misma tensión máxima equivalente a la que se produce en la novena revolución para $m_1$. ¿Cuál es la relación entre las masas?
    \item A la revolución anterior, ambas masas se sueltan y describen un movimiento parabólico. Entregue la condición para que $m_1$ observe a $m_2$ en reposo para la dirección horizontal.
\end{enumerate}
\newpage
\item Los bloques $M_1$ y $M_2$ de masa con el mismo nombre, se deslizan hacia abajo de un plano inclinado fijo de ángulo $\theta$ como se indica en la figura. El bloque $M_2$ esta apoyado (no pegado) sobre el bloque $M_1$. Los coeficientes de roce dinámico entre el bloque $M_1$ y el plano y entre el bloque $M_2$ y el plano valen $\mu_1$ y $\mu_2$ respectivamente. Determine la aceleración común de los dos bloques y la magnitud de la fuerza $F_1$ que $M_2$ ejerce sobre $M_1$. Discuta los movimientos posibles del conjunto y las condiciones que se deben cumplir para que exista la fuerza $\vec{F_1}$.

\begin{figure}[H]
    \centering
    \svgpath{../../2023-1/img/aux_4}
    \hspace{1em}
    \includesvg[width=0.8\textwidth]{aux_4_p2.svg}
\end{figure}


% Para imágenes vectoriales -> el texto tiene que estar en LaTeX
% \begin{figure}[htbp]
%   \centering
%   \svgpath{../Imagenes/ejercicios}  -> .. irse pa'trás 
%   \includesvg{ej5.svg}
% \end{figure}

\end{enumerate}
\end{document}

\documentclass[letterpaper,11pt]{article}
\oddsidemargin -1.0cm \textwidth 17.5cm

\usepackage[utf8]{inputenc}
\usepackage[activeacute,spanish, es-lcroman]{babel}
\decimalpoint
\usepackage{amsfonts,setspace}
\usepackage{amsmath}
\usepackage{amssymb, amsmath, amsthm}
\usepackage{comment}
\usepackage{float}
\usepackage{amssymb}
\usepackage{dsfont}
\usepackage{anysize}
\usepackage{multicol}
\usepackage{enumerate}
\usepackage{graphicx}
\usepackage[left=1.5cm,top=2cm,right=1.5cm, bottom=1.7cm]{geometry}
\setlength\headheight{1.5em} 
\usepackage{fancyhdr}
\usepackage{multicol}
\usepackage{hyperref}
\usepackage{wrapfig}
\usepackage{subcaption}
\usepackage{siunitx}
\usepackage{cancel}
\usepackage{mdwlist}
\usepackage{svg}
\pagestyle{fancy}
\fancyhf{}
\renewcommand{\labelenumi}{\normalsize\bfseries P\arabic{enumi}.}
\renewcommand{\labelenumii}{\normalsize\bfseries (\alph{enumii})}
\renewcommand{\labelenumiii}{\normalsize\bfseries \roman{enumiii})}


\begin{document}

\fancyhead[L]{\itshape{Facultad de Ciencias F\'isicas y Matem\'aticas}}
\fancyhead[R]{\itshape{Universidad de Chile}}
\rfoot[]{pág. \thepage}

\begin{minipage}{11.5cm}
    \begin{flushleft}
        \hspace*{-0.6cm}\textbf{FI1000-1 Introducción a la Física Clásica}\\
        \hspace*{-0.6cm}\textbf{Profesor:} Ignacio Bordeu\\
        \hspace*{-0.6cm}\textbf{Auxiliares:} Alejandro Cartes \& Simón Yáñez\\
        \hspace*{-0.6cm}\textbf{Ayudante:} Javier Cubillos\\
    \end{flushleft}
\end{minipage}

\begin{picture}(2,3)
    \put(366, 10){\includegraphics[scale=0.9]{2020-1/Imágenes/logo/dfi-fcfm.pdf}}
\end{picture}

\begin{center}
	\LARGE\textbf{Auxiliar \#10}\\
	\Large{Conservación de Momentum Lineal}
\end{center}

\vspace{-1cm}
\begin{enumerate}\setlength{\itemsep}{0.4cm}

\item[]

\item Un proyectil de masa $m_0$ se mueve a una velocidad $v_0$ y explota en dos fragmentos de masas $m_a$ y $m_b$, respectivamente. Inmediatamente después de la explosión se mueven en las direcciones indicadas. ¿Cuál es la velocidad de cada fragmento?

\begin{figure}[htbp]
  \centering
  \svgpath{../../2023-1/img/aux_10}
  \includesvg[width=0.65\linewidth]{Aux 10 - P1.svg}
\end{figure}

\item Un bloque de masa $m$ se dirige hacia la derecha a velocidad $v_a$ por una superficie lisa. Colisiona con otro bloque de masa $M$ que se dirige a la izquierda a velocidad $v_b$. La colisión es \textbf{inelástica} y el coeficiente de restitución es $r = \frac{3}{4}$.

\begin{enumerate}
    \item Determinar las velocidades de los bloques después de la colisión.
    \item Hallar la pérdida de energía en la colisión.
    \item Resolver a) desde el punto de vista del centro de masas.
\end{enumerate}
\begin{figure}[H]
  \centering
  \svgpath{../../2023-1/img/aux_10}
  \includesvg[width=0.65\linewidth]{Aux 10 - P2.svg}
\end{figure}

\item Considere un péndulo consistente de una masa $m$ colgada de un hilo de largo $L$. Suponga que el péndulo inicialmente parte con el hilo en posición horizontal. Al llegar la masa al punto inferior (punto \textbf{O} de la figura), choca elásticamente con una masa $M = 2m$ que se mueve con velocidad $v_0$ como indica la figura. El péndulo rebota (hacia atrás) llegando a su misma posición inicial.

\begin{figure}[htbp]
  \centering
  \svgpath{../../2023-1/img/aux_10}
  \includesvg[width=0.7\linewidth]{Aux 10 - P3.svg}
\end{figure}

\begin{enumerate}
    \item Encuentre la rapidez inicial $v_0$ en función de $m$, $M$, $L$ y $g$.
    \item Encuentre la velocidad de $M$ después del choque.
\end{enumerate}

\item \textbf{[C3 - Otoño 2022]} Considere una nave espacial de masa $M$ y un atronauta de masa $m$ (incluye traje y equipo), inicialmente en reposo con el astronauta dentro de la nave. En un cierto instante el astronauta debe salir de la nave y se impulsa hacia la derecha con rapidez $v_0$.

\begin{figure}[htbp]
  \centering
  \svgpath{../../2023-1/img/aux_10}
  \includesvg[width=0.7\linewidth]{Aux 10 - P4.svg}
\end{figure}

\begin{enumerate}
    \item Calcula la rapidez con la que la nave espacial se mueve luego que el astronauta la abandona.
    \item Luego que el astronauta se ha movido una cierta distancia de la nave, decide volver. Si el astronauta tiene una pistola con una única bala de masa $m_b$ que puede disparar con rapidez $v_b$ con respecto al astronauta, calcule la masa $m_b$ de tal forma que el astronauta pueda volver a su nave. Comente respecto a los casos límite en que $v_b \gg v_0$ y $v_b = v_0$.
\end{enumerate}
    


% Para imágenes vectoriales -> el texto tiene que estar en LaTeX
% \begin{figure}[htbp]
%   \centering
%   \svgpath{../Imagenes/ejercicios}  -> .. irse pa'trás 
%   \includesvg{ej5.svg}
% \end{figure}

\end{enumerate}
\end{document}
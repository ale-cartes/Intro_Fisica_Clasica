\documentclass[letterpaper,11pt]{article}
\oddsidemargin -1.0cm \textwidth 17.5cm

\usepackage[utf8]{inputenc}
\usepackage[activeacute,spanish, es-lcroman]{babel}
\decimalpoint
\usepackage{amsfonts,setspace}
\usepackage{amsmath}
\usepackage{amssymb, amsmath, amsthm}
\usepackage{comment}
\usepackage{float}
\usepackage{amssymb}
\usepackage{dsfont}
\usepackage{anysize}
\usepackage{multicol}
\usepackage{enumerate}
\usepackage{graphicx}
\usepackage[left=1.5cm,top=2cm,right=1.5cm, bottom=1.7cm]{geometry}
\setlength\headheight{1.5em} 
\usepackage{fancyhdr}
\usepackage{multicol}
\usepackage{hyperref}
\usepackage{wrapfig}
\usepackage{subcaption}
\usepackage{siunitx}
\usepackage{cancel}
\usepackage{mdwlist}
\usepackage{svg}
\pagestyle{fancy}
\fancyhf{}
\renewcommand{\labelenumi}{\normalsize\bfseries P\arabic{enumi}.}
\renewcommand{\labelenumii}{\normalsize\bfseries (\alph{enumii})}
\renewcommand{\labelenumiii}{\normalsize\bfseries \roman{enumiii})}


\begin{document}

\fancyhead[L]{\itshape{Facultad de Ciencias F\'isicas y Matem\'aticas}}
\fancyhead[R]{\itshape{Universidad de Chile}}
\rfoot[]{pág. \thepage}

\begin{minipage}{11.5cm}
    \begin{flushleft}
        \hspace*{-0.6cm}\textbf{FI1000-1 Introducción a la Física Clásica}\\
        \hspace*{-0.6cm}\textbf{Profesor:} Ignacio Bordeu\\
        \hspace*{-0.6cm}\textbf{Auxiliares:} Alejandro Cartes \& Simón Yáñez\\
        \hspace*{-0.6cm}\textbf{Ayudante:} Javier Cubillos\\
    \end{flushleft}
\end{minipage}

\begin{picture}(2,3)
    \put(366, 10){\includegraphics[scale=0.9]{2020-1/Imágenes/logo/dfi-fcfm.pdf}}
\end{picture}

\begin{center}
	\LARGE\textbf{Auxiliar \#9}\\
	\Large{Trabajo y Energía}
\end{center}

\vspace{-1cm}
\begin{enumerate}\setlength{\itemsep}{0.4cm}

\item[]

\item Un bloque de masa $m$ se desliza por un plano inclinado sin roce de largo $L$ con un ángulo de inclinación~$\theta$ con respecto a la horizontal. Al final de este plano, el bloque se mueve por una superficie áspera con coeficiente de roce cinético $\mu_c$. Determine la distancia $D$ a la cual el bloque se detendrá en esta superficie áspera.

\begin{figure}[htbp]
  \centering
  \svgpath{../../2021-1/Imagenes/ejercicios}
  \includesvg[width=0.65\linewidth]{ej7.svg}
\end{figure}

\item Un bloque de masa $m$ choca con un bloque de masa $M$ en una superficie horizontal con coeficiente de roce cinético $\mu_m$ y $\mu_M$ respectivo a cada masa. Tras la colisión los bloques viajan juntos, pero no pegados, a una rapidez desconocida.
\begin{enumerate}
    \item Si la distancia recorrida por el sistema es $d$, determine la rapidez inicial
    \item El trabajo realizado por $F_{Mm}$ y $F_{mM}$
\end{enumerate}

\item Un bloque de masa $m$ es eyectado por un resorte de constante elástica $k$. El bloque desliza por un tramo horizontal con roce y luego sube por un tramo lista una altura $H$ por determinar. Una vez que el cubo alcanza el punto más alto, este retorna y comprime el resorte en $\beta\Delta$, con $\Delta$ la compresión del resorte utilizada para eyectar el bloque. Determina la altura máxima $H$ de subida del bloque en el tramo indicado.

\begin{figure}[H]
    \centering
    \svgpath{../../2021-2/img/aux9}
    \includesvg[width=0.8\linewidth]{p2.svg}
\end{figure}



% Para imágenes vectoriales -> el texto tiene que estar en LaTeX
% \begin{figure}[htbp]
%   \centering
%   \svgpath{../Imagenes/ejercicios}  -> .. irse pa'trás 
%   \includesvg{ej5.svg}
% \end{figure}

\end{enumerate}
\end{document}

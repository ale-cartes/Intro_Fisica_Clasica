\documentclass[letterpaper,11pt]{article}
\oddsidemargin -1.0cm \textwidth 17.5cm

\usepackage[utf8]{inputenc}
\usepackage[activeacute,spanish, es-lcroman]{babel}
\decimalpoint
\usepackage{amsfonts,setspace}
\usepackage{amsmath}
\usepackage{amssymb, amsmath, amsthm}
\usepackage{comment}
\usepackage{float}
\usepackage{amssymb}
\usepackage{dsfont}
\usepackage{anysize}
\usepackage{multicol}
\usepackage{enumerate}
\usepackage{graphicx}
\usepackage[left=1.5cm,top=2cm,right=1.5cm, bottom=1.7cm]{geometry}
\setlength\headheight{1.5em} 
\usepackage{fancyhdr}
\usepackage{multicol}
\usepackage{hyperref}
\usepackage{wrapfig}
\usepackage{subcaption}
\usepackage{siunitx}
\usepackage{cancel}
\usepackage{mdwlist}
\usepackage{svg}
\pagestyle{fancy}
\fancyhf{}
\renewcommand{\labelenumi}{\normalsize\bfseries P\arabic{enumi}.}
\renewcommand{\labelenumii}{\normalsize\bfseries (\alph{enumii})}
\renewcommand{\labelenumiii}{\normalsize\bfseries \roman{enumiii})}


\begin{document}

\fancyhead[L]{\itshape{Facultad de Ciencias F\'isicas y Matem\'aticas}}
\fancyhead[R]{\itshape{Universidad de Chile}}

\begin{minipage}{11.5cm}
    \begin{flushleft}
        \hspace*{-0.6cm}\textbf{FI1000-1 Introducción a la Física Clásica}\\
        \hspace*{-0.6cm}\textbf{Profesor:} Ignacio Bordeu\\
        \hspace*{-0.6cm}\textbf{Auxiliares:} Alejandro Cartes \& Simón Yáñez\\
        \hspace*{-0.6cm}\textbf{Ayudante:} Javier Cubillos\\
    \end{flushleft}
\end{minipage}

\begin{picture}(2,3)
    \put(366, 10){\includegraphics[scale=0.9]{2020-1/Imágenes/logo/dfi-fcfm.pdf}}
\end{picture}

\begin{center}
	\LARGE\textbf{Auxiliar \#2}\\
	\Large{Cinemática 1D y 2D}
\end{center}

\vspace{-1cm}
\begin{enumerate}\setlength{\itemsep}{0.4cm}

\rfoot[]{pág. \thepage}

\item[]

\item  \textbf{[\textit{C1 - Otoño 2022}]} Era la noche más oscura en Gotham, la luna comenzaba a esconderse en el horizonte
y el plan maestro de dos súper criminales iba a ser perpetrado. El plan consistía en asaltar
uno de los más grandes bancos de la ciudad, cuyas arcas estaban compuestas por grandes
fortunas de diferentes multimillonarios. Luego de perpetrar exitosamente el robo, el líder y
su cómplice se disponían a huir rumbo a un auto que los esperaba frente al asilo Arkham,
muchos metros más allá. En $t = 0$, el líder comenzó a correr con una velocidad de $v_l$ hacia la
derecha, mientras que su cómplice, mucho más lento por su falta de forma, a una velocidad
$v_c$ ($v_l > v_c$), también hacia la derecha. En un tiempo $t = T$, luego del aviso desesperado
de Jim Gordon, Batman aterriza heroicamente en el banco, con su capa flameante al viento
y rodilla al piso. Este se abalanza sobre los ladrones, partiendo desde el banco y desde el
reposo, con una aceleración constante $a_b$. Tras una vertiginosa persecución, Batman captura
fácilmente al cómplice, quien le suplicaba perdón, mientras que en ese mismo instante lanza
un grito ensordecedor al líder, ordenándole detenerse. Un fuerte escalofrío recorrió la espalda
del líder, afectando sus piernas, desacelerando con aceleración constante y deteniéndose luego
de avanzar una distancia $d_0$ desde que escuchó el grito.


\begin{enumerate}
    \item Represente, en un mismo gráfico, la posición de ambos malhechores y de Batman en
función del tiempo. Indique claramente cuál curva representa al líder de los malhechores,
cuál representa a su cómplice, y cuál representa a Batman.
    \item ¿A qué tiempo $t_2$ Batman logra alcanzar al cómplice, desde que los malhechores
comienzan a escapar?
    \item ¿A qué distancia del banco se encuentran Batman y cada uno de los malhechores
cuando Batman da alcance al cómplice?
    \item ¿Qué aceleración $a$ tiene el líder luego de que que Batman le ordena detenerse?
    \item ¿A qué tiempo $t_3$ se detuvo el líder, desde que comenzó su escapatoria?
\end{enumerate}

\newpage

\item Michael Scott quiere probar si puede saltar en una cama elástica desde un edificio sin hacerse daño. Para aquello lanza una sandía con velocidad $v_1$ en un ángulo $\theta$ desde el techo de Dunder Mifflin Inc. Prontamente se da cuenta que la caída es sumamente peligrosa cuando observa la sandía salir eyectada con velocidad $v_2$ y ángulo $\alpha$ desde la cama elástica destruyendo el auto de Stanley. ($H,h_s,l_1$ y $h_a$ también son datos del problema)

\begin{figure}[H]
    \centering
        \centering
        \svgpath{../../2023-1/img/aux_2}
        \includesvg[width=0.8\textwidth]{Dunder_Mifflin.svg}
\end{figure}

\begin{enumerate}
    \item Encuentre el tiempo $t_1$ en el que la sandía toca la cama elástica y la velocidad $v_1$ con la que fue lanzada. 
    \item Despeje una expresión para $\alpha$ sabiendo que $v_2 = v(t_1)$ en función de los datos del problema. ($v(t_1)$ es la velocidad final del primer tramo)
    \item Finalmente, encuentre la distancia desde el impacto con el trampolín al auto de Stanley.
\end{enumerate}


% Para imágenes vectoriales -> el texto tiene que estar en LaTeX
% \begin{figure}[htbp]
%   \centering
%   \svgpath{../Imagenes/ejercicios}  -> .. irse pa'trás 
%   \includesvg{ej5.svg}
% \end{figure}

\end{enumerate}
\end{document}

\documentclass[letterpaper,11pt]{article}
\oddsidemargin -1.0cm \textwidth 17.5cm

\usepackage[utf8]{inputenc}
\usepackage[activeacute,spanish, es-lcroman]{babel}
\decimalpoint
\usepackage{amsfonts,setspace}
\usepackage{amsmath}
\usepackage{amssymb, amsmath, amsthm}
\usepackage{comment}
\usepackage{float}
\usepackage{amssymb}
\usepackage{dsfont}
\usepackage{anysize}
\usepackage{multicol}
\usepackage{enumerate}
\usepackage{graphicx}
\usepackage[left=1.5cm,top=2cm,right=1.5cm, bottom=1.7cm]{geometry}
\setlength\headheight{1.5em} 
\usepackage{fancyhdr}
\usepackage{multicol}
\usepackage{hyperref}
\usepackage{wrapfig}
\usepackage{subcaption}
\usepackage{siunitx}
\usepackage{cancel}
\usepackage{mdwlist}
\usepackage{svg}
\pagestyle{fancy}
\fancyhf{}
\renewcommand{\labelenumi}{\normalsize\bfseries P\arabic{enumi}.}
\renewcommand{\labelenumii}{\normalsize\bfseries (\alph{enumii})}
\renewcommand{\labelenumiii}{\normalsize\bfseries \roman{enumiii})}


\begin{document}

\fancyhead[L]{\itshape{Facultad de Ciencias F\'isicas y Matem\'aticas}}
\fancyhead[R]{\itshape{Universidad de Chile}}

\begin{minipage}{11.5cm}
    \begin{flushleft}
        \hspace*{-0.6cm}\textbf{FI1000-1 Introducción a la Física Clásica}\\
        \hspace*{-0.6cm}\textbf{Profesor:} Ignacio Bordeu\\
        \hspace*{-0.6cm}\textbf{Auxiliares:} Alejandro Cartes \& Simón Yáñez\\
        \hspace*{-0.6cm}\textbf{Ayudante:} Javier Cubillos\\
    \end{flushleft}
\end{minipage}

\begin{picture}(2,3)
    \put(366, 10){\includegraphics[scale=0.9]{2020-1/Imágenes/logo/dfi-fcfm.pdf}}
\end{picture}

\begin{center}
	\LARGE\textbf{Ejercicio \#2}
\end{center}

\vspace{-1cm}
\begin{enumerate}\setlength{\itemsep}{0.4cm}

\rfoot[]{pág. \thepage}

\item[]

% \item \large{\textit{\textbf{AC: Propuesta de ej!}}} Un auto se encuentra viajando con rapidez $v_0$ constante. Tras haber recorrido una distancia $L$, el conductor decide frenar con una aceleración constante $a_0$. Determine la distancia a la cual se detiene y el tiempo que le tomó

\item Un automóvil viaja por la Ruta 5 Norte a exceso de velocidad, con rapidez constante $v_a$. El automóvil no se detiene en un punto de control policial, por lo que el carabinero ubicado en el control decide iniciar una persecución.

El carabinero tarda un tiempo $\tau$ en subirse a su vehículo, desde que pasa el automóvil por el control. Una vez en su vehículo, el carabinero acelera con magnitud constante $a_p$, hasta dar con el automóvil infractor.

\begin{enumerate}
    \item (1.5 puntos) Represente cualitativamente, en un único gráfico, las posiciones del automóvil infractor y vehículo policial en función del tiempo.
    \item (1.5 puntos) Determine la distancia entre del automóvil infractor y el vehículo policial, en el instante en que el policía comienza a acelerar.
    \item (1.5 puntos) Determine el tiempo que tarda el policía en alcanzar al automóvil infractor, desde el momento en que el automóvil infractor pasa por el control policial.
    \item (1.5 puntos) Determine la distancia total recorrida por el vehículo policial hasta dar con el vehículo infractor.
\end{enumerate}

\hline

% Para imágenes vectoriales -> el texto tiene que estar en LaTeX
% \begin{figure}[htbp]
%   \centering
%   \svgpath{../Imagenes/ejercicios}  -> .. irse pa'trás 
%   \includesvg{ej5.svg}
% \end{figure}

\end{enumerate}
\end{document}

\documentclass[letterpaper,11pt]{article}
\oddsidemargin -1.0cm \textwidth 17.5cm

\usepackage[utf8]{inputenc}
\usepackage[activeacute,spanish, es-lcroman]{babel}
\decimalpoint
\usepackage{amsfonts,setspace}
\usepackage{amsmath}
\usepackage{amssymb, amsmath, amsthm}
\usepackage{comment}
\usepackage{float}
\usepackage{amssymb}
\usepackage{dsfont}
\usepackage{anysize}
\usepackage{multicol}
\usepackage{enumerate}
\usepackage{graphicx}
\usepackage[left=1.5cm,top=2cm,right=1.5cm, bottom=1.7cm]{geometry}
\setlength\headheight{1.5em} 
\usepackage{fancyhdr}
\usepackage{multicol}
\usepackage{hyperref}
\usepackage{wrapfig}
\usepackage{subcaption}
\usepackage{siunitx}
\usepackage{cancel}
\usepackage{mdwlist}
\usepackage{svg}
\pagestyle{fancy}
\fancyhf{}
\renewcommand{\labelenumi}{\normalsize\bfseries P\arabic{enumi}.}
\renewcommand{\labelenumii}{\normalsize\bfseries (\alph{enumii})}
\renewcommand{\labelenumiii}{\normalsize\bfseries \roman{enumiii})}


\begin{document}

\fancyhead[L]{\itshape{Facultad de Ciencias F\'isicas y Matem\'aticas}}
\fancyhead[R]{\itshape{Universidad de Chile}}

\begin{minipage}{11.5cm}
    \begin{flushleft}
        \hspace*{-0.6cm}\textbf{FI1000-1 Introducción a la Física Clásica}\\
        \hspace*{-0.6cm}\textbf{Profesora:} Paulina Lira\\
        \hspace*{-0.6cm}\textbf{Auxiliares:} Alejandro Cartes \& Juan Cristóbal Castro\\
        \hspace*{-0.6cm}\textbf{Ayudantes:} Francisca Bórquez \& Catalina Molina\\
    \end{flushleft}
\end{minipage}

\begin{picture}(2,3)
    \put(366, 10){\includegraphics[scale=0.9]{2020-1/Imágenes/logo/dfi-fcfm.pdf}}
\end{picture}

\begin{center}
	\LARGE\textbf{Auxiliar \#8}\\
	\Large{Roce}
\end{center}

\vspace{-1cm}
\begin{enumerate}\setlength{\itemsep}{0.4cm}

\rfoot[]{pág. \thepage}

\item[]

\item Determine el máximo valor que puede tener $m_3$ para que $m_1$ no se caiga si el coeficiente de fricción estático entre $m_1$ y $m_2$ es $\mu_e$, y el de fricción cinemático entre $m_2$ y la mesa es $\mu_c$

\begin{figure}[h]
    \centering
    \svgpath{../img/aux8/}
    \includesvg[width=0.39\linewidth]{p1a8.svg}
\end{figure}

\item Una carretera está peraltada de modo que un automóvil, desplazándose a una rapidez $v=$ \SI{80}{\km/\hour}, puede tomar la curva de $R=$ \SI{30}{m} de radio, incluso si existe una capa de hielo equivalente a un coeficiente de fricción aproximadamente cero. Con esto determine el ángulo de inclinación $\theta$.

Determine el intervalo de velocidades a que un automóvil puede tomar esta curva sin patinar si los coeficientes de roce estático y cinemático, entre la carretera y las ruedas, son $\mu_e = 0.3$ y $\mu_c = 0.26$ respectivamente.

\item Sobre un plano inclinado liso, que forma un ángulo $\theta$ con la horizontal, se desliza un bloque partiendo del reposo. Después de recorrer una distancia $D$, el bloque entra en un tramo rugoso. El bloque se detiene luego de recorrer una distancia $D$ en dicho tramo. Calcule el coeficiente de roce (¿cinético o estático?) entre el bloque y la superficie rugosa.

\begin{figure}[H]
    \centering
    \svgpath{../../2021-1/Imagenes/aux7/}
    \begin{subfigure}[t]{0.45\textwidth}
        \includesvg[width=0.9\linewidth]{p2.svg}
        \caption*{Figura P2}
    \end{subfigure}
    \hspace{0.3em}
    \begin{subfigure}[t]{0.45\textwidth}
        \includesvg[width=0.9\linewidth]{preg.svg}
        \caption*{Figura P3}
    \end{subfigure}
    % \begin{subfigure}[t]{0.40\textwidth}
    %     \includesvg[width=0.85\linewidth]{p3.svg}
    %     \caption*{Figura P3}
    % \end{subfigure}
\end{figure}

\item \textbf{[Propuesto]} ¿Cuál es el máximo valor que puede tener $m_3$ para que $m_1$ no se caiga? Considere el coeficiente de roce estático entre $m_1$ y $m_2$ igual a $\mu_e$ y los dos coeficientes de roce cinemático (entre $m_2$ y el plano horizontal, y entre $m_3$ y el plano inclinado) iguales a $\mu_c$.

\begin{figure}[H]
    \centering
    \svgpath{../../2021-1/Imagenes/aux7/}
    \includesvg[width=0.4\linewidth]{p3.svg}
\end{figure}

% Para imágenes vectoriales -> el texto tiene que estar en LaTeX
% \begin{figure}[htbp]
%   \centering
%   \svgpath{../Imagenes/ejercicios}  -> .. irse pa'trás 
%   \includesvg{ej5.svg}
% \end{figure}

\end{enumerate}
\end{document}

\documentclass[letterpaper,11pt]{article}
\oddsidemargin -1.0cm \textwidth 17.5cm

\usepackage[utf8]{inputenc}
\usepackage[activeacute,spanish, es-lcroman]{babel}
\decimalpoint
\usepackage{amsfonts,setspace}
\usepackage{amsmath}
\usepackage{amssymb, amsmath, amsthm}
\usepackage{comment}
\usepackage{float}
\usepackage{amssymb}
\usepackage{dsfont}
\usepackage{anysize}
\usepackage{multicol}
\usepackage{enumerate}
\usepackage{graphicx}
\usepackage[left=1.5cm,top=2cm,right=1.5cm, bottom=1.7cm]{geometry}
\setlength\headheight{1.5em} 
\usepackage{fancyhdr}
\usepackage{multicol}
\usepackage{hyperref}
\usepackage{wrapfig}
\usepackage{subcaption}
\usepackage{siunitx}
\usepackage{cancel}
\usepackage{mdwlist}
\usepackage{svg}
\pagestyle{fancy}
\fancyhf{}
\renewcommand{\labelenumi}{\normalsize\bfseries P\arabic{enumi}.}
\renewcommand{\labelenumii}{\normalsize\bfseries (\alph{enumii})}
\renewcommand{\labelenumiii}{\normalsize\bfseries \roman{enumiii})}


\begin{document}

\fancyhead[L]{\itshape{Facultad de Ciencias F\'isicas y Matem\'aticas}}
\fancyhead[R]{\itshape{Universidad de Chile}}

\begin{minipage}{11.5cm}
    \begin{flushleft}
        \hspace*{-0.6cm}\textbf{FI1000-6 Introducción a la Física Clásica}\\
        \hspace*{-0.6cm}\textbf{Profesora:} Paulina Lira\\
        \hspace*{-0.6cm}\textbf{Auxiliares:} Juan Cristóbal Castro \& Alejandro Silva\\
        \hspace*{-0.6cm}\textbf{Ayudantes:} Francisca Bórquez, Catalina Molina \& Erick Pérez\\
        
    \end{flushleft}
\end{minipage}

\begin{picture}(2,3)
    \put(366, 10){\includegraphics[scale=0.9]{2020-1/Imágenes/logo/dfi-fcfm.pdf}}
\end{picture}

\begin{center}
	\LARGE\textbf{Auxiliar \#9}\\
	\Large{Fuerza elástica y Gravitación}
\end{center}

\vspace{-1cm}
\begin{enumerate}\setlength{\itemsep}{0.4cm}

\rfoot[]{pág. \thepage}

\item[]

\item En presencia de la gravedad terrestre, una bolita de masa $m$ es sostenida mediante un resorte de constante elástica $k$ y longitud natural $L$. El conjunto se dispone dentro de un tubo de paredes lisas inclinado en un ángulo $\beta$ con respecto a la vertical. El tubo se hace girar con velocidad angular constante $\omega$ y la bolita mantiene una trayectoria circunferencial. El extremo superior $Q$ del resorte se ubica en el eje de rotación. Determine la elongación $\Delta$ del resorte y discuta la posibilidad de que $\Delta = 0$

\item Cuatro partículas idénticas de masa $m$ se unen mediante resortes idénticos de masa nula, constante elástica $k$ y longitud natural $L$. El sistema toma la forma cuadrada de la figura mientras rota en torno a su centro con velocidad angular $\omega$. Calcule la elongación experimentada por los resortes.

\begin{figure}[H]
    \centering
    \svgpath{../Imagenes/aux9/}
    \begin{subfigure}[t]{0.45\textwidth}
        \centering
        \includesvg[width=0.8\linewidth]{resorte1.svg}
        \caption*{Figura P1}
    \end{subfigure}
    \hspace{0.1em}
    \begin{subfigure}[t]{0.45\textwidth}
        \centering
        \includesvg[width=0.5\linewidth]{cuadrao.svg}
        \caption*{Figura P2}
    \end{subfigure}
\end{figure}


\item Dos planetas de masas $m_1$ y $m_2$ giran en órbitas circulares de radio $r_1$ y $r_2$ respectivamente, en torno a un astro de masa $M >> m_1 \& m_2$. Un resorte de constante elástica $k$ une ambos planetas haciendo que giren con el mismo periodo. Despreciando la interacción gravitacional entre las masas pequeñas, determine una expresión para $k$. Considere que el resorte posee masa despreciable y largo natural nulo.

\item Tres satélites idénticos de masa $m$ experimentan órbitas circunferenciales de igual radio $R$ cuando se ordenan en una configuración triangular equilátera. Al centro de las órbitas se ubica un planeta de masa $M$. Sin despreciar la interacción gravitacional entre los satélites, determine la rapidez con que estos orbitan.

\begin{figure}[H]
    \centering
    \svgpath{../Imagenes/aux9/}
    \begin{subfigure}[t]{0.45\textwidth}
        \centering
        \includesvg[width=0.55\linewidth]{resorte-planeta.svg}
        \caption*{Figura P3}
    \end{subfigure}
    \hspace{0.1em}
    \begin{subfigure}[t]{0.45\textwidth}
        \centering
        \includesvg[width=0.65\linewidth]{equilat.svg}
        \caption*{Figura P4}
    \end{subfigure}
\end{figure}



\end{enumerate}
\end{document}
